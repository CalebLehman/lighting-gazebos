\documentclass[11pt]{article}
\usepackage{amsmath,amssymb,amsthm}
\usepackage[pdftex]{graphicx}
\usepackage{geometry}
\geometry{margin=0.5in}
\usepackage{asymptote}

\begin{document}

\section*{Minimal-Overlap Covering of a Hexagonal Gazebo}

Given a graph $G = (V, E)$ and edge weights $w: E \to \mathbb{R}_{> 0}$,
it is natural to ask for a minimum-weight path containing all edges.
We describe an equivalent reformulation to this problem
and apply it to the very real and important problem of covering the edges of a particular hexagonal gazebo as efficiently as possible.

\subsection*{Reformulation}

We define a \emph{path} on $G$ to be a sequence $p = v_0, e_1, v_1, \ldots, e_k, v_k$ such that
$e_i$ is an edge from $v_{i-1}$ to $v_{i}$ for each $i$.
We define a \emph{complete} path on $G$ to be a path $p = v_0, e_1, v_1, \ldots, e_k, v_k$ such that
$E \subseteq \{e_1, \ldots, e_k\}$.
We can extend the weight function to paths by defining $w(p) = \sum_{i} w(e_i)$.
In this terminology, our original problem is to find a complete path on $p$ which minimizes $w(p)$.

We call a path \emph{non-overlapping} when $e_i = e_j$ if and only if $i = j$
and define an \emph{extension} $G' = (V, E')$ of $G$ to be the original graph with some number of the existing edges duplicated $1$ or more times.
Note that any complete path $p$ on $G$
induces an equal-weight, complete, non-overlapping path on the extension $G'$
obtained by duplicating each edge in $G$ to match the number of times it appears in $p$.
Conversely, any complete, non-overlapping path $p$ on an extension $G'$
induces an equal-weight, complete path on $G$.
Hence, our original problem is equivalent to finding the minimal weight
over all complete, non-overlapping paths on any extension of $G$.
The weight of a complete, non-overlapping path over a graph is simply the sum of all edge weights,
so our final, equivalent reformulation is \emph{to find the minimal edge-weight over extensions of $G$ which admit a complete, non-overlapping path}.

\subsection*{Gazebo Application}

We wish to hang holiday lights on the edges of a hexagonal gazebo roof with a single strand of lights while minimizing waste.
The roof and its relevant distances are modeled by the following graph with the obvious rotational symmetries:

\begin{center}
\begin{asy}
    include olympiad;
    size(7cm); 
    defaultpen(1.5);

    pair O = (0, 0);
    pair A = dir(0);
    pair B = dir(60); 
    pair C = dir(120); 
    pair D = dir(180); 
    pair E = dir(240);
    pair F = dir(300);

    draw(A--D, black); 
    draw(B--E, black); 
    draw(C--F, black); 
    draw(A--B--C--D--E--F--cycle, black);
    label("$x$", A--B);
    label("$y$", A--O);

    label("$O$", O, 4N);
    label("$A$", A, dir(A));
    label("$B$", B, dir(B));
    label("$C$", C, dir(C));
    label("$D$", D, dir(D));
    label("$E$", E, dir(E));
    label("$F$", F, dir(F));
\end{asy}
\end{center}
Note that, when projected flat, $\triangle{AOB}$ is equilateral,
so $y \geq x$, with equality only in the case of a degenerate, flat roof.
Any extension of this graph admitting a complete, non-overlapping path can have at most two vertices of odd degree\footnote{%
    Vertices in the middle of a complete, non-overlapping path are entered exactly as many times as they are exited,
    so only the start and end vertices can have odd degree%
}.
The above graph contains $6$ vertices with an odd degree,
and each duplicated edge affects the degree of exactly $2$ vertices,
so the desired extension must have at least $\frac{6-2}{2} = 2$ duplicated edges.
Since $x < y$,
the minimal possible weight added by these edges is $2 x$.
It follows that all extensions which admit a complete, non-overlapping path have a total edge weight of at least $6 x + 6 y + 2 x = 8 x + 6 y$.
From the equivalent formulation given above,
the length of a light strand covering all edges is therefore bounded below by $8 x + 6 y$.
The path determined by the sequence of vertices $A \to O \to B \to C \to O \to D \to E \to O \to F \to A \to B \to C \to D \to E \to F$
has length $y + y + x + y + y + x + y + y + x + x + x + x + x + x = 8 x + 6 y$.
Since it is achievable, $8 x + 6 y$ is the minimal possible length necessary to cover the gazebo edges.

\end{document}
